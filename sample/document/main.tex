\documentclass[a4j,uplatex]{jsarticle}

\usepackage{float}
\usepackage[dvipdfmx]{hyperref,graphicx}

% for hyperref(cf: https://ossyaritoori.hatenablog.com/entry/2016/11/11/%E3%80%90Tex%E3%80%91%E5%8D%92%E8%AB%96%E4%BF%AE%E8%AB%96%E3%81%AB%E3%81%AF%E6%98%AF%E9%9D%9E%E3%83%8F%E3%82%A4%E3%83%91%E3%83%BC%E3%83%AA%E3%83%B3%E3%82%AF%E6%A9%9F%E8%83%BD%E3%82%92%E3%81%A4)
\usepackage{pxjahyper}
\hypersetup{
	colorlinks=false, % リンクに色をつけない設定
	bookmarks=true, % 以下ブックマークに関する設定
	bookmarksnumbered=true,
	pdfborder={0 0 0},
	bookmarkstype=toc,
}
\newcommand{\linedhref}[2]{\underline{\href{#1}{\emph{#2}}}}

\begin{document}

吾輩は猫である.名前はまだ無い\cite{Soseki1905}.
どこで生れたかとんと見当がつかぬ.
何でも薄暗いじめじめした所で
ニャーニャー泣いていた事だけは記憶している.
吾輩はここで始めて人間というものを見た.

\begin{figure}[H]
	\begin{center}
		\includegraphics[height=5cm]{img/natsume_soseki.jpeg}
		\caption{夏目漱石の写真(引用元:\linedhref{https://ja.wikipedia.org/wiki/\%E5\%A4\%8F\%E7\%9B\%AE\%E6\%BC\%B1\%E7\%9F\%B3}{夏目漱石 - Wikipedia})}
	\end{center}
\end{figure}

\bibliography{ref}
\bibliographystyle{junsrt}

\end{document}